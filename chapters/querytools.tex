\chapter{Querying Tools}
\label{querytools}

In the previous chapter, we discussed the proof-of-concept implementation of MM-quecat, which exists in the form of a Python library.
However, despite its interface being easy to use, it does not fully leverage the power of MMQL with respect to the end user experience.
Recall that in~\cref{category:section:querylanguage}, we articulated the requirements that MMQL \textit{be expressive and readable}, and that it \textit{be intuitive and familiar to users of existing query languages}.
While we believe that MMQL meets these design goals as proposed, and that one of its greatest strengths lies in being graphically expressive, meaning users can define graph patterns which visually resemble the actual structures being defined.
Therefore, in this chapter, we present the prototype of a user interface (UI) application for MM-quecat\footnote{\url{https://github.com/yawnston/quecat-frontend}}, which allows the user to visually construct queries.

The presented prototype is not one of the main products of this thesis, but we present it regardless, as it should give the reader an idea of how the language may be used from an end-user perspective.
The prototype of the UI for MM-quecat was created by the author of this thesis while working on a not-yet-published demo paper~\cite{mm_quecat}, co-authored with Pavel Koupil and Irena Holubov{\'a}.
Therefore its implementation will be completed as the paper nears its publishing date, and in this chapter, we present it as it exists in its current form.

The querying tool is a web application built with React\footnote{\url{https://reactjs.org/}} and Next.js\footnote{\url{https://nextjs.org/}}, utilizing the Cytoscape.js\footnote{\url{https://js.cytoscape.org/}} library for graph visualization and MUI\footnote{\url{https://mui.com/}} as a component library.

\section{Requirements}

Before we introduce our design, we first need to formalize the requirements for such an application.
The application should primarily serve as a querying tool, meaning it should be possible for the user to construct a query using the schema category as a visual aid.
The user should have the ability to directly execute the query, and retrieve the results in the chosen representation (like JSON for example).

Aside from this primary use case, the user should also have the ability to view all of the possible query plans created by MM-quecat for the execution of the query.
Each query plan should be able to be examined further, displaying the native database queries generated by MM-quecat for this specific query plan, as well as visualizing the query plan in a subset of the full schema category.

As a whole, the query tool should heavily leverage the graphical nature of MMQL, nicely visualizing the query in the schema category and possibly even providing semantic syntax highlighting in the query itself, with colors corresponding to the concepts visualized in the schema category.

\section{User Interface}

Now that we have formalized the requirements for a MMQL query tool, we will showcase our prototype.
The main query screen can be seen in~\cref{fig:uiquery}, where on the right side, the user may construct an MMQL query, and on the right side, a visual aid in the form of the schema category is present.
To help the user understand the query that they are writing, the schema category display also contains a visualization of various query features.

\begin{figure}[h]
\centering
\shadowimage[width=.95\textwidth]{img/quecat-ui-query.png} 
\caption{MM-quecat querying UI, showing the query on the right, and the schema category with the visualized query on the left.}
\label{fig:uiquery}
\end{figure}

For example, the \texttt{Order} schema object is colored blue, as it lies on the query path, but is not a part of the projection.
The \texttt{Contact} object is colored gray since it is not part of the query at all, and the \texttt{Name} object corresponding to a \texttt{Product} is colored purple, since it is part of the query projection.
Lastly, we can see that the \texttt{Name} object corresponding to a \texttt{Customer} is colored purple with an orange outline, meaning that it is part of the projection, and there exists a filter on this schema object.

If the application user clicks on the "EXPLAIN" button in the bottom part of the screen, they will be taken to the screen shown in~\cref{fig:uiplans}.
There they can see a list of all possible query plans, coupled with their plan cost and the databases used in the plan.

\begin{figure}[h]
\centering
\shadowimage[width=.95\textwidth]{img/quecat-ui-plans.png} 
\caption{A table displaying all query plans generated by MM-quecat and their details.}
\label{fig:uiplans}
\end{figure}

The user may further examine individual query plans, leading to the screen shown in~\cref{fig:uidetail}.
This screen, similarly to the screen shown in~\cref{fig:uiquery}, shows the schema category on the left side, however in this instance, the schema category is restricted to the databases and schema objects which are part of this specific query plan.
On the right side, the user can see the native database queries which were generated for this plan by MM-quecat.
The user can use this information to verify that the queries generated by MM-quecat are, for instance, using the correct indexes in the corresponding databases.

\begin{figure}[h]
\centering
\shadowimage[width=.95\textwidth]{img/quecat-ui-detail.png} 
\caption{Detailed view of a specific query plan, showing the native database queries generated by MM-quecat.}
\label{fig:uidetail}
\end{figure}
