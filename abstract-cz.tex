%%% A template for a simple PDF/A file like a stand-alone abstract of the thesis.

\begin{filecontents*}[overwrite]{\jobname.xmpdata}
    \Author{Daniel Crha}
    \Title{Unifikované dotazování nad multi-modelovými daty}
    \Publisher{Univerzita Karlova}
\end{filecontents*}


\documentclass[12pt]{report}

\usepackage[a4paper, hmargin=1in, vmargin=1in]{geometry}
\usepackage[a-2u]{pdfx}
\usepackage[utf8]{inputenc}
\usepackage[T1]{fontenc}
\usepackage{lmodern}
\usepackage{textcomp}

\begin{document}

%% Do not forget to edit abstract.xmpdata.

Drtivá většina současných multi-model dotazovacích řešení vyžaduje, aby uživatel měl rozsáhlé znalosti použitých datových modelů. Existuje jeden přístup pro unifikované multi-model dotazování, ale tento přístup není prakticky použitelný pro většinu uživatelů, protože je velmi komplexní. Tato práce představuje MMQL, což je multi-model dotazovací jazyk založený na teorii kategorií, který byl inspirován dotazovacím jazykem SPARQL. Za použití MMQL mohou uživatelé formulovat multi-model, multi-databázové dotazy, aniž by museli vědět o způsobu uložení dat. Dále tato práce představuje návrh na postup implementace MMQL, včetně podpůrných algoritmů. Pro ověření validity tohoto návrhu také obsahuje jeho základní implementaci ve formě nástroje MM-quecat. Tento nástroj byl experimentálně ověřen ve scénáři zahrnujícím PostgreSQL a MongoDB, přičemž obě databáze byly unifikovaně dotazovány pomocí jednoho MMQL dotazu. Jelikož se jedná o jeden z prvních přístupů pro unifikované multi-model dotazování, tato práce dále analyzuje slabiny a omezení navrženého přístupu, což umožní lépe cílit navazující práci v této oblasti.

\end{document}
